\section{Py Serial}

	\subsection{Python}
	Python adalah bahasa pemrograman yang dibuat oleh Guido van Rossum dan populer sebagai bahasa pemrograman scripting dan Web. Mengacu pada ide wikipedia, Python adalah bahasa pemrograman interpretatif multiguna dengan filosofi desain yang berfokus pada keterbacaan kode. 
	Python dikenal sebagai bahasa pemrograman yang menggabungkan nilai nilai kapabilitas, kemampuan, dengan sintaks kode yang sangat begitu jelas, dan dilengkapi dengan fungsi penyimpanan standar yang menjadikannya komprehensif dan komprehensif. 
	Python mendukung pemrograman multi-paradigma, khususnya, tetapi tidak terbatas, pada pemrograman berorientasi objek, pemrograman imperatif, dan pemrograman fungsional. 
	Python memiliki salah satu fitur yang sangat unik yaitu sebagai bahasa pemrograman dinamis yaitu yang datang dengan manajemen memori otomatis. Seperti bahasa 
	
\subsection{dua cara menjalankan python}
  
Python dapat dijalankan dengan dua cara, yakni:
1. Mode command-line
	dengan mode ini, program dapat dilakukan dengan memanggil interpreter, kemudian memberi statement Python dan interpreter akan menampilkan hasil, sebagai contoh:
bash-2.05a$ python
Python 2.1.3 (#1, Apr 20 2002, 10:14:34)
[GCC 2.95.4 20011002 (Debian prerelease)] on linux2 type "copyright", "credits" or "license" for more information.
>>> print 1+1
2
baris pertama dalam contoh diatas adalah perintah untuk memanggil interpreter python. dua baris kemudian merupakan pesan dari interpreter yang menunjukan versi dan informasi copyright, credit, dan lisensinya. Sedangkan baris keempat merupakan prompt interpreter yang menunjukan bahwa interpreter siap menerima input. dalam contoh diatas diberi input print 1+1, interpreter menghasilkan 2.
2. Mode script
	mode ini dilakukan dengan cara menuliskan keseluruhan program dalam file, kemudian interpreter akan mengeksekusi seluruh isi dalam file. File seperti ini dinamakan script. sebagai contoh, sebuah file yang ditulis dengan text editor dan diberi nama script01.py memiliki isi sebagai berikut:
print 1+1
Dengan perjanjian bahwa file yang berisi program Python diberi nama dengan ekstensi .py. Eksekusi dari program dengan mode ini dapat dilakukan dengan cara berikut:
$ python script01.py
2
contoh:
# -------------
#  Program Sederhana
#   by Ema & Wawan
#--------------
print "I Like Python"
Catatan: Python tidak membutuhkan titik koma (semicolon) dibelakang statement. Semicolon disini sifatnya optional sehingga bisa digunakan, tetapi juga boleh tidak dipakai.
